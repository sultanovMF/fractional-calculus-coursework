\documentclass[a4paper, fontsize=14pt]{article}
\usepackage{scrextend}
\usepackage{indentfirst, fancyhdr, amsfonts, mathtools, amssymb}
\usepackage{titlesec} %работа с рубрикацией
\usepackage{tocloft} %настройки оглавления
\usepackage[T2A]{fontenc}
\usepackage[utf8x]{inputenc}
\usepackage[russian]{babel}
\usepackage{hyperref} %кликабельное оглавление
\usepackage[left=3.7cm,right=2cm,top=2cm,bottom=2cm]{geometry}
\usepackage{tempora} %настраиваем шрифт типа TNR                                   
\usepackage{newtxmath} %делаем шрифт формул похожим на TNR
\usepackage[labelsep=endash]{caption}
\usepackage{listings}
\usepackage{ucs}
\usepackage{amsmath}
\usepackage{pdfpages}
\usepackage{tocloft}
\usepackage{textcase}
\usepackage{textcomp}
\renewcommand\cftsecafterpnum{\vskip15pt}
\lstset{
  columns=fullflexible,
  breaklines=true,
}
\linespread{1}
\setcounter{page}{1} %в зависимости от того, какой по счёту страницей должно быть оглавление!

%НАСТРОЙКИ ОГЛАВЛЕНИЯ

\renewcommand{\cftsecaftersnum}{.} %точки после номеров разделов и подразделов в оглавлении
\renewcommand{\cftsubsecaftersnum}{.}
\renewcommand{\cftsecfont}{\normalfont} %разделы в оглавлении пишутся обычным (не жирным) шрифтом
\renewcommand{\cftsecpagefont}{\normalfont} %соответствующие им страницы тоже
\renewcommand{\cftsecleader}{\cftdotfill{\cftdotsep}} %расставляем точки между названиями разделов и их страницами
\addto\captionsrussian{\renewcommand\contentsname{СОДЕРЖАНИЕ}} %хотим, чтобы слово "Содержание" писалось капсом
\renewcommand{\cfttoctitlefont}{\hfil\bfseries} %слово СОДЕРЖАНИЕ по центру жирным
\renewcommand{\cftaftertoctitle}{\hfill}
% \renewcommand{\cftsecfont}{\MakeUppercase}

\captionsetup[figure]{name=Рисунок}
%НАСТРОЙКИ РУБРИКАЦИИ
\titleformat*{\section}{\center\bf} %названия разделов и подразделов по середине жирным шрифтом
\titleformat*{\subsection}{\center\bf}
\titlelabel{\thetitle.\quad} %название раздела и его номер отделены точкой

%НАСТРОЙКИ БИБЛИОГРАФИИ
\addto\captionsrussian{\renewcommand\refname{СПИСОК ЛИТЕРАТУРЫ}} %хотим, чтобы слова "Список литературы" писались капсом
\makeatletter
\renewcommand{\@biblabel}[1]{#1.} %хотим, чтобы в списке литературы номера источников писались в формате "No. <...>", а не "[No] <...>"
\makeatother
\newtheorem{definition}{Определение}

% НАСТРОЙКИ ПРОИЗВОДНЫХ
\newcommand{\MD}[1]{\mathbb{D}_{#1}^{\alpha}} % Marchaud derivative
\newcommand{\RLD}[2]{{}_{#1}\mathcal{D}_{#2}^{\alpha}} % Riemann Liuville derivative
\newcommand{\RLI}[2]{{}_{#1}\mathcal{I}_{#2}^{\alpha}} % Riemann Liuville integral



\begin{document}
% \includepdf[pages={1-3}]{src/word pdfs.pdf} % титульник
\tableofcontents

\newpage
\section*{ВВЕДЕНИЕ}
\addcontentsline{toc}{section}{Введение}

\section[Некоторые сведения из группового анализа]{НЕКОТОРЫЕ СВЕДЕНИЯ ИЗ ГРУППОВОГО АНАЛИЗА}

\section[Дробные производные Римана-Лиувилля]{ДРОБНЫЕ ПРОИЗВОДНЫЕ РИМАНА-ЛИУВИЛЛЯ}
$\RLD{a}{x}$
\section[Дробные производные Маршо]{ДРОБНЫЕ ПРОИЗВОДНЫЕ МАРШО???????????}

\begin{equation}
  \MD{ +} \psi=\frac{\left\{\alpha\right\}}{\Gamma(1-\left\{\alpha\right\})}\int_{0}^{\infty}\frac{f^{(n)}(x)-f^{(n)}(x-\xi)}{\xi^{1+\left(\alpha\right)}}d\xi
\end{equation}
\begin{equation}
  \MD{ -} \psi=\frac{\{\alpha\}}{\Gamma(1-\{\alpha\})}\int_{0}^{\infty}\frac{f^{(n)}(x)-f^{(n)}(x+\xi)}{\xi^{1+\{\alpha\}}\ } d\xi
\end{equation}

Здесь $n=[\alpha], \alpha = n + \{ \alpha \}$. При $\alpha \in (1, 2)$ имеем $n=1$. 
% ⠀⣤⣤⣤⣤⣤⣤⣤⣤⣤⣤⣤⣤⣤⣤⣤⣤⣤⣤⣤⣤⣤⣤⡄⡀⠀⠀⣤⣤⣤⣤⣤⣤⣤⣤⣤⣤⣤⣤⣤⣤⣤⣤⣤⣤⣤⣤⣤⣤⡄
% ⠀⢸⣿⣿⣿⣿⣿⣿⣿⣿⣿⣿⣿⣿⣿⣿⣿⣿⣿⣿⣿⣿⣿⣷⠠⠀⠰⣿⣿⣿⣿⣿⣿⣿⣿⣿⣿⣿⣿⣿⣿⣿⣿⣿⣿⣿⣿⣿⣿⠁
% ⠀⣿⣿⣿⣿⣿⣿⣿⣿⣿⣿⣿⣿⣿⣿⣿⣿⣿⣿⣿⣿⣿⣿⣿⠀⠀⢠⣿⣿⣿⣿⣿⣿⣿⣿⣿⣿⣿⣿⣿⣿⣿⣿⣿⣿⣿⣿⣿⡏⠀
% ⢸⣿⣿⣿⣿⣿⣿⣿⣿⣿⣿⣿⣿⣿⣿⣿⣿⣿⣿⣿⣿⣿⣿⣧⠋⠀⢀⣿⣿⣿⣿⣿⣿⣿⣿⣿⣿⣿⣿⣿⣿⣿⣿⣿⣿⣿⣿⣿⣿⡄
% ⢸⣿⣿⣿⣿⣿⣿⣿⣿⣿⣿⣿⣿⣿⣿⣿⣿⣿⣿⣿⣿⣿⣿⣿⠒⠀⢸⣿⣿⣿⣿⣿⣿⣿⣿⣿⣿⣿⣿⣿⣿⣿⣿⣿⣿⣿⣿⣿⣿⡇
% ⢸⣿⣿⣿⣿⣿⣿⣿⣿⣿⣿⣿⣿⣿⣿⣿⣿⣿⣿⣿⣿⣿⣿⡏⡤⠀⠸⣿⣿⣿⣿⣿⣿⣿⣿⣿⣿⣿⣿⣿⣿⣿⣿⣿⣿⣿⣿⣿⣿⡇
% ⢸⣿⣿⣿⣿⣿⣿⣿⣿⣿⣿⣿⣿⣿⣿⣿⣿⣿⣿⣿⣿⣿⣿⡿⣠⠀⢸⣿⣿⣿⣿⣿⣿⣿⣿⣿⣿⣿⣿⣿⣿⣿⣿⣿⣿⣿⣿⣿⣿⡇
% ⢸⣿⣿⣿⣿⣿⣿⣿⣿⣿⣿⣿⣿⣿⣿⣿⣿⣿⣿⣿⣿⣿⣿⣿⢀⠀⢸⣿⣿⣿⣿⣿⣿⣿⣿⣿⣿⣿⣿⣿⣿⣿⣿⣿⣿⣿⣿⣿⣿⡇
% ⢸⣿⣿⣿⣿⣿⣿⣿⣿⣿⣿⣿⣿⣿⣿⣿⣿⣿⣿⣿⣿⣿⣿⣯⠁⠀⢸⣿⣿⣿⣿⣿⣿⣿⣿⣿⣿⣿⣿⣿⣿⣿⣿⣿⣿⣿⣿⣿⣿⡇
% ⢸⣿⣿⣿⣿⣿⣿⣿⣿⣿⣿⣿⣿⣿⣿⣿⣿⣿⣿⣿⣿⣿⣿⡏⠊⠀⢸⣿⣿⣿⣿⣿⣿⣿⣿⣿⣿⣿⣿⣿⣿⣿⣿⣿⣿⣿⣿⣿⣿⡇
% ⢸⣿⣿⣿⣿⣿⣿⣿⣿⣿⣿⣿⣿⣿⣿⣿⣿⣿⣿⣿⣿⣿⣿⡇⠖⠀⠸⢿⣿⣿⣿⣿⣿⣿⣿⣿⣿⣿⣿⣿⣿⣿⣿⣿⣿⣿⣿⣿⣿⡇
% ⢸⣿⣿⣿⣿⣿⣿⣿⣿⣿⣿⣿⣿⣿⣿⣿⣿⣿⣿⣿⣿⡿⣵⣇⣠⠂⠀⠀⢹⣿⣿⣿⣿⣿⣿⣿⣿⣿⣿⣿⣿⣿⣿⣿⣿⣿⣿⣿⣿⡇
% ⢸⣿⣿⣿⣿⣿⣿⣿⣿⣿⣿⣿⣿⣿⣿⣿⣿⣿⣿⣿⠿⠣⡋⠛⠛⠗⠀⠀⢠⣿⣿⣿⣿⣿⣿⣿⣿⣿⣿⣿⣿⣿⣿⣿⣿⣿⣿⣿⣿⡇
% ⢸⣿⣿⣿⣿⣿⣿⣿⣿⣿⣿⣿⣿⣿⣿⣿⣿⣿⣿⡏⢘⣧⣿⣦⣀⠀⠀⠀⠀⢹⣿⣿⣿⣿⣿⣿⣿⣿⣿⣿⣿⣿⣿⣿⣿⣿⣿⣿⣿⡇
% ⢸⣿⣿⣿⣿⣿⣿⣿⣿⣿⣿⣿⣿⣿⣿⣿⣿⣿⣿⡟⠻⣉⠓⣤⡈⠁⠀⠀⠀⣼⣿⣿⣿⣿⣿⣿⣿⣿⣿⣿⣿⣿⣿⣿⣿⣿⣿⣿⣿⡇
% ⢸⣿⣿⣿⣿⣿⣿⣿⣿⣿⣿⣿⣿⣿⣿⣿⣿⣿⡿⠓⠶⡞⠛⠦⠈⠠⠀⠀⢀⣿⣿⣿⣿⣿⣿⣿⣿⣿⣿⣿⣿⣿⣿⣿⣿⣿⣿⣿⣿⡇
% ⢸⣿⣿⣿⣿⣿⣿⣿⣿⣿⣿⣿⣿⣿⣿⣿⣿⣿⣷⣌⢃⣽⣗⣆⡈⢀⠀⠀⠸⣿⣿⣿⣿⣿⣿⣿⣿⣿⣿⣿⣿⣿⣿⣿⣿⣿⣿⣿⣿⡇
% ⢸⣿⣿⣿⣿⣿⣿⣿⣿⣿⣿⣿⣿⣿⣿⣿⣿⣿⣿⡈⢦⡩⣧⢈⠀⠄⠀⠀⠸⣿⣿⣿⣿⣿⣿⣿⣿⣿⣿⣿⣿⣿⣿⣿⣿⣿⣿⣿⣿⡇
% ⢸⣿⣿⣿⣿⣿⣿⣿⣿⣿⣿⣿⣿⣿⣿⣿⣿⣿⣿⠫⠹⣏⣙⠛⢅⠀⠀⠀⠐⣿⣿⣿⣿⣿⣿⣿⣿⣿⣿⣿⣿⣿⣿⣿⣿⣿⣿⣿⣿⡇
% ⢸⣿⣿⣿⣿⣿⣿⣿⣿⣿⣿⣿⣿⣿⣿⣿⣿⣿⣿⣦⣤⣌⢉⡀⠀⠑⡀⠀⠀⣿⣿⣿⣿⣿⣿⣿⣿⣿⣿⣿⣿⣿⣿⣿⣿⣿⣿⣿⣿⡇
% ⢸⣿⣿⣿⣿⣿⣿⣿⣿⣿⣿⣿⣿⣿⣿⣿⣿⣿⣿⣿⣿⢏⠜⠀⣠⡇⡀⠰⢾⣿⣿⣿⣿⣿⣿⣿⣿⣿⣿⣿⣿⣿⣿⣿⣿⣿⣿⣿⣿⡇
% ⢸⣿⣿⣿⣿⣿⣿⣿⣿⣿⣿⣿⣿⣿⣿⣿⣿⣿⣿⣿⢫⡆⠀⣰⣿⣾⡇⠀⠘⣿⣿⣿⣿⣿⣿⣿⣿⣿⣿⣿⣿⣿⣿⣿⣿⣿⣿⣿⣿⡇
% ⢸⣿⣿⣿⣿⣿⣿⣿⣿⣿⣿⣿⣿⣿⣿⣿⣿⣿⣿⢫⠄⢀⣼⣿⣿⣿⣧⣾⠀⠈⢿⣿⣿⣿⣿⣿⣿⣿⣿⣿⣿⣿⣿⣿⣿⣿⣿⣿⣿⡇
% ⢸⣿⣿⣿⣿⣿⣿⣿⣿⣿⣿⣿⣿⣿⣿⣿⣿⡿⢁⠄⢀⣼⣿⣿⣿⣿⣿⣷⢰⠀⠀⠻⣿⣿⣿⣿⣿⣿⣿⣿⣿⣿⣿⣿⣿⣿⣿⣿⣿⡇
% ⢸⣿⣿⣿⣿⣿⣿⣿⣿⣿⣿⣿⣿⣿⣿⣿⡿⠡⠄⢀⣼⣿⣿⣿⣿⣿⣿⣿⣿⣄⡆⠀⠙⣿⣿⣿⣿⣿⣿⣿⣿⣿⣿⣿⣿⣿⣿⣿⣿⡇
% ⢸⣿⣿⣿⣿⣿⣿⣿⣿⣿⣿⣿⣿⣿⣿⣿⢳⠖⠀⣾⣿⣿⣿⣿⣿⣿⣿⣿⣿⣿⣥⣴⠀⠙⣿⣿⣿⣿⣿⣿⣿⣿⣿⣿⣿⣿⣿⣿⣿⡇
% ⢸⣿⣿⣿⣿⣿⣿⣿⣿⣿⣿⣿⣿⣿⣿⠃⠖⠀⣼⣿⣿⣿⣿⣿⣿⣿⣿⣿⣿⣿⣿⣯⣠⠀⠘⣿⣿⣿⣿⣿⣿⣿⣿⣿⣿⣿⣿⣿⣿⡇
% ⢸⣿⣿⣿⣿⣿⣿⣿⣿⣿⣿⣿⣿⣿⡏⠞⠀⣰⣿⣿⣿⣿⣿⣿⣿⣿⣿⣿⣿⣿⣿⣿⣯⣠⠀⠹⣿⣿⣿⣿⣿⣿⣿⣿⣿⣿⣿⣿⣿⡇
% ⢸⣿⣿⣿⣿⣿⣿⣿⣿⣿⣿⣿⣿⣿⠡⠆⠀⣿⣿⣿⣿⣿⣿⣿⣿⣿⣿⣿⣿⣿⣿⣿⣿⣟⡀⠀⢻⣿⣿⣿⣿⣿⣿⣿⣿⣿⣿⣿⣿⡇
% ⢸⣿⣿⣿⣿⣿⣿⣿⣿⣿⣿⣿⣿⣿⠔⠀⢸⣿⣿⣿⣿⣿⣿⣿⣿⣿⣿⣿⣿⣿⣿⣿⣿⣿⠃⠀⢸⣿⣿⣿⣿⣿⣿⣿⣿⣿⣿⣿⣿⡇
% ⢸⣿⣿⣿⣿⣿⣿⣿⣿⣿⣿⣿⣿⣿⠖⠀⢸⣿⣿⣿⣿⣿⣿⣿⣿⣿⣿⣿⣿⣿⣿⣿⣿⣿⠚⠀⢸⣿⣿⣿⣿⣿⣿⣿⣿⣿⣿⣿⣿⡇
% ⠘⣿⣿⣿⣿⣿⣿⣿⣿⣿⣿⣿⣿⣿⣠⠀⠀⣿⣿⣿⣿⣿⣿⣿⣿⣿⣿⣿⣿⣿⣿⣿⣿⣿⠖⠀⢸⣿⣿⣿⣿⣿⣿⣿⣿⣿⣿⣿⣿⠇
% ⠠⡧⠹⣿⣿⣿⣿⣿⣿⣿⣿⣿⣿⣿⣧⡄⠀⠸⢿⣿⣿⣿⣿⣿⣿⣿⣿⣿⣿⣿⣿⣿⣿⠱⠂⠀⣾⣿⣿⣿⣿⣿⣿⣿⣿⣿⣿⡏⣿⠀
% ⠀⣿⣄⡘⠿⣿⣿⣿⣿⣿⣿⣿⣿⣿⣿⣧⡆⠀⠈⠻⣿⣿⣿⣿⣿⣿⣿⣿⣿⣿⣿⢛⡴⠛⢀⣾⣿⣿⣿⣿⣿⣿⣿⣿⣿⣿⣿⣣⣿⠀
% ⠀⠀⠀⣿⠀⣿⣿⣿⣿⣿⣿⣿⣿⣿⣿⣿⣷⣼⡀⡄⠈⠉⠻⠿⠿⠿⠿⢟⣛⡉⠴⠋⢁⣰⣿⣿⣿⣿⣿⣿⣿⣿⣿⣿⣿⣿⣿⣿⣿⠀
% ⠀⠀⣴⣿⣷⣿⣿⣿⣿⣿⣿⣿⣿⣿⣿⣿⣿⣿⣿⣿⣌⣢⡐⠄⠐⠤⠕⠒⠈⣉⣤⣶⣿⣿⣿⣿⣿⣿⣿⣿⣿⣿⣿⣿⣿⣿⣿⣿⡏⠀
% ⠀⠀⢽⣿⣿⣿⣿⣿⣿⣿⣿⣿⣿⣿⣾⣿⣿⣯⣿⣿⣽⣿⣟⣷⣾⣷⣾⣿⣿⣿⣿⣿⣿⣿⣿⣿⣿⣿⣿⣿⣿⣿⣿⣿⠟⢻⣿⣿⡇⠀
% ⠀⠀⠀⠀⠉⠉⠉⠋⠉⠛⠛⠛⠛⠛⠛⠛⠛⠛⠛⠛⠛⠛⠛⠛⠛⠛⠛⠛⠛⠛⠛⠛⠿⠿⠿⠿⠿⠿⠿⠿⠿⠛⠛⠛⠁⠀⠙⠛⠁⠀

\section[Обобщение уравнения Шрёдингера с производными Римана-Лиувилля]{ОБОБЩЕНИЕ УРАВНЕНИЯ ШРЁДИНГЕРА С ПРОИЗВОДНЫМИ РИМАНА-ЛИУВИЛЛЯ}


\section[Обобщение уравнения  Шрёдингера с производными Маршо]{ОБОБЩЕНИЕ УРАВНЕНИЯ ШРЁДИНГЕРА С ПРОИЗВОДНЫМИ МАРШО??????????}


\newpage
\section*{ЗАКЛЮЧЕНИЕ}
\addcontentsline{toc}{section}{Заключение}
В ходе выполнения курсовой работы был изучен метод численного решения волнового уравнения и уравнения затухающих колебаний
с помощью явной разностной схемы с краевыми условиями Мура и отражения. Показано, что поглощение по условиям Мура 
выполняется корректно в случае колебания с затуханием. В качестве примера комбинации отражающих и поглощающих условий была рассмотрена дифракция на двух щелях.

Также был смоделирован переход волнового пакета из одной однородной среды в другую и продемонстрировано разложение волны на преломленную
с меньшей амплитудой и отраженную, которая сложилась с проходящей волной. Был программно реализовано затухание колебательного процесса, 
а также его применимость к построению идеально согласованных слоев (PML).

\newpage

\addcontentsline{toc}{section}{Список литературы}
\begin{thebibliography}{9}
  \bibitem{Chen}
  Chen, Jingyi Zhao, Jian-Guo.,
  Application of the Nearly Perfectly Matched Layer to Seismic-Wave Propagation Modeling in Elastic Anisotropic Media. //
  The Bulletin of the Seismological Society of America. — 2011 — V.101. — P. 2866-2871.
  \bibitem{Mur}
  Mur, G., Absorbing boundary conditions for the finite-difference approximation of the time-domain electromagnetic-fieldequations //
   IEEE Trans. Electromagn. Compat. — 1981 — Vol. EMC-23, No. 4,— P. 377–382.
  \bibitem{berenger} 
  J. Berenger, A perfectly matched layer for the absorption of electromagnetic waves [Текст] / J. Berenger //
  Journal of Computational Physics. — 1994. — V. 114. — P. 185-200
  \bibitem{Loh}
  P.-R. Loh, A. F. Oskooi, M. Ibanescu, M. Skorobogatiy, S. G. Johnson.,
  Fundamental relation between phase and group velocity, and application
  to the failure of perfectly matched layers in backward-wave structures [Текст] //
  Physical Review E. — 2009. — V. 79, — P. 13-24.
  \bibitem{ChenHsuZhou} 
  Chen G., Hsu S., Zhou J., Snapback repellers as a cause of chaotic vibration of the wave equation
  with a van der Pol boundary condition and energy injection at the middle of the span [Текст] // Journal of
  Mathematical Physics — 1998. — V.39 — P. 6459–6489
  \bibitem{Johnson}
  S. G. Johnson, Notes on the algebraic structure of wave equations. [Текст] /  S. G. Johnson // Journal of Mathematical Physics — 2001. — P. 18
  \bibitem{kalitkin}
  Калиткин Н.Н. Численные методы [Текст]: учебник / Н.Н. Калиткин. — Москва: Наука, 1978. — 512 с.
  \bibitem{samarskiy}
  Самарский А.А. Теория разностных схем [Текст]: учебник / А.А.Самарский, — Москва: Наука, 1977. — 656 с.
  \end{thebibliography}

 % \includepdf[pages={17-31}]{src/word pdfs.pdf}
\end{document}


